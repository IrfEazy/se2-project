% A LaTeX template for MSc Thesis submissions to 
% Politecnico di Milano (PoliMi) - School of Industrial and Information Engineering
%
% S. Bonetti, A. Gruttadauria, G. Mescolini, A. Zingaro
% e-mail: template-tesi-ingind@polimi.it
%
% Last Revision: October 2021
%
% Copyright 2021 Politecnico di Milano, Italy. NC-BY

\documentclass{Configuration_Files/PoliMi3i_thesis}

%------------------------------------------------------------------------------
%	REQUIRED PACKAGES AND  CONFIGURATIONS
%------------------------------------------------------------------------------

% CONFIGURATIONS
\usepackage{parskip} % For paragraph layout
\usepackage{setspace} % For using single or double spacing
\usepackage{emptypage} % To insert empty pages
\usepackage{multicol} % To write in multiple columns (executive summary)
\setlength\columnsep{15pt} % Column separation in executive summary
\setlength\parindent{0pt} % Indentation
\raggedbottom

% PACKAGES FOR TITLES
\usepackage{titlesec}
% \titlespacing{\section}{left spacing}{before spacing}{after spacing}
\titlespacing{\section}{0pt}{3.3ex}{2ex}
\titlespacing{\subsection}{0pt}{3.3ex}{1.65ex}
\titlespacing{\subsubsection}{0pt}{3.3ex}{1ex}
\usepackage{color}

% PACKAGES FOR LANGUAGE AND FONT
\usepackage[english]{babel} % The document is in English  
\usepackage[utf8]{inputenc} % UTF8 encoding
\usepackage[T1]{fontenc} % Font encoding
\usepackage[11pt]{moresize} % Big fonts

% PACKAGES FOR IMAGES
\usepackage{graphicx}
\usepackage{transparent} % Enables transparent images
\usepackage{eso-pic} % For the background picture on the title page
\usepackage{subfig} % Numbered and caption subfigures using \subfloat.
\usepackage{tikz} % A package for high-quality hand-made figures.
\usetikzlibrary{}
\graphicspath{{./Images/}} % Directory of the images
\usepackage{amsthm} % Coloured "Theorem"
\usepackage{thmtools}
\usepackage{xcolor}
\usepackage{float}

% STANDARD MATH PACKAGES
\usepackage{amsmath}
\usepackage{amssymb}
\usepackage{amsfonts}
\usepackage{bm}
\usepackage[overload]{empheq} % For braced-style systems of equations.
\usepackage{fix-cm} % To override original LaTeX restrictions on sizes

% PACKAGES FOR TABLES
\usepackage{tabularx}
\usepackage{longtable} % Tables that can span several pages
\usepackage{colortbl}

% PACKAGES FOR ALGORITHMS (PSEUDO-CODE)
\usepackage{algorithm}
\usepackage{algorithmic}

% PACKAGES FOR REFERENCES & BIBLIOGRAPHY
\usepackage[
    colorlinks=true,
    linkcolor=black,
    anchorcolor=black,
    citecolor=black,
    filecolor=black,
    menucolor=black,
    runcolor=black,
    urlcolor=black
]{hyperref} % Adds clickable links at references
\usepackage{cleveref}
\usepackage[square, numbers, sort&compress]{natbib} % Square brackets, citing references with numbers, citations sorted by appearance in the text and compressed
\bibliographystyle{abbrvnat} % You may use a different style adapted to your field

% OTHER PACKAGES
\usepackage{pdfpages} % To include a pdf file
\usepackage{afterpage}
\usepackage{lipsum} % DUMMY PACKAGE
\usepackage{fancyhdr}
\usepackage{wasysym} % For the headers
\usepackage{rotating}

\usepackage{listings}
\input{Configuration_Files/alloy.sty}
\lstdefinestyle{alloy}{
    commentstyle=\itshape,
    keywordstyle=\bfseries,
    stringstyle=\itshape,
}

%\definecolor{dkgreen}{rgb}{0,0.6,0}
%\definecolor{gray}{rgb}{0.5,0.5,0.5}
%\definecolor{mauve}{rgb}{0.58,0,0.82}

%\lstset{frame=tb,
%    language=Alloy,
%    aboveskip=3mm,
%    belowskip=3mm,
%    showstringspaces=false,
%    columns=flexible,
%    basicstyle={\small\ttfamily},
%    numbers=none,
%    numberstyle=\tiny\color{gray},
%    keywordstyle=\color{blue},
%    commentstyle=\color{dkgreen},
%    stringstyle=\color{mauve},
%    breaklines=true,
%    breakatwhitespace=true,
%    tabsize=3
%}



\fancyhf{}

% Input of configuration file. Do not change config.tex file unless you really know what you are doing. 
\input{Configuration_Files/config}

%----------------------------------------------------------------------------
%	NEW COMMANDS DEFINED
%----------------------------------------------------------------------------

%----------------------------------------------------------------------------
%	ADD YOUR PACKAGES (be careful of package interaction)
%----------------------------------------------------------------------------

%----------------------------------------------------------------------------
%	ADD YOUR DEFINITIONS AND COMMANDS (be careful of existing commands)
%----------------------------------------------------------------------------

%----------------------------------------------------------------------------
%	BEGIN OF YOUR DOCUMENT
%----------------------------------------------------------------------------

\begin{document}
    \fancypagestyle{plain}{%
        \fancyhf{} % Clear all header and footer fields
        \fancyhead[RO,RE]{\thepage} %RO=right odd, RE=right even
        \renewcommand{\headrulewidth}{0pt}
        \renewcommand{\footrulewidth}{0pt}}

%----------------------------------------------------------------------------
%	TITLE PAGE
%----------------------------------------------------------------------------

    \pagestyle{empty} % No page numbers
    \frontmatter % Use roman page numbering style (i, ii, iii, iv...) for the preamble pages

    \puttitle{
        title=Software Engineering 2\\Requirements Analysis and\\Specification Document,
        name1=Irfan Cela - 10694934, % Author Name and Surname
        name2=Mario Cela - 10685242,
        name3=Alessandro Cogollo - 10571078,
        academicyear=2022-2023,
    } % These info will be put into your Title page

%----------------------------------------------------------------------------
%	PREAMBLE PAGES: ABSTRACT (inglese e italiano), EXECUTIVE SUMMARY
%----------------------------------------------------------------------------
    \startpreamble
    \setcounter{page}{1} % Set page counter to 1

%----------------------------------------------------------------------------
%	LIST OF CONTENTS/FIGURES/TABLES/SYMBOLS
%----------------------------------------------------------------------------

% TABLE OF CONTENTS
    \thispagestyle{empty}
    \tableofcontents % Table of contents
    \thispagestyle{empty}
    \cleardoublepage

%-------------------------------------------------------------------------
%	THESIS MAIN TEXT
%-------------------------------------------------------------------------
% In the main text of your thesis you can write the chapters in two different ways:
%
%(1) As presented in this template you can write:
%    \chapter{Title of the chapter}
%    *body of the chapter*
%
%(2) You can write your chapter in a separated .tex file and then include it in the main file with the following command:
%    \chapter{Title of the chapter}
%    \input{chapter_file.tex}
%
% Especially for long thesis, we recommend you the second option.

    \addtocontents{toc}{\vspace{2em}} % Add a gap in the Contents, for aesthetics
    \mainmatter % Begin numeric (1,2,3...) page numbering


    \chapter{Introduction}
    \label{ch:introduction}%
    \section{Purpose}
\label{sec:purpose}%
Climate change is a topical issue for everybody these days, and it's leading a change not only in our habits but also in consumes, which implies a change in the production model our society adopted until now.
Private mobility is one of the market sectors which is changing the most, directly dependent from fossil fuel.

In order to keep global warming below 1.5°C, Europe have decided to reduce greenhouse gas emissions of CO2 per
person per year by 2030, and, by the same year, the IEA predicts that electric vehicles will have a market share of roughly 30 percent, with a total number of 23 million e-cars on the roads: electric vehicles represent eco-friendly mobility solutions that are, and will be on our roads in the next future.

Most of the current electric cars can travel between 150 and 350 kilometers on a single charge, but premium-brand models can currently cover more than 500 kilometers.
This being said, it's obvious that, when people use an electric vehicle, knowing where to charge it and carefully planning the
charging process in such a way that it introduces minimal interference and constraints on our daily schedule
is of great importance.

That's were \verb|eMALL| operates: it can find charging stations owned by several Charging Point Operators - CPO - and,
considering the activities in user's schedule, it can propose the best possible path of charging process
in order to minimize the cost and the waisted time at the station.


\section{Scope}
\label{sec:scope}%
%TODO: complete scope section.


\section{Definition, Acronyms, Abbreviations}
\label{sec:definition_acronyms_abbreviations}%
\begin{table}[H]
    \begin{center}
        \begin{tabular}{ |l|l| }
            \hline
            \textbf{Acronyms} & \textbf{Definition}            \\
            \hline
            eMSP              & e-Mobility Service Provider    \\
            \hline
            CPO               & Charging Point Operator        \\
            \hline
            CPMS              & Charge Point Management System \\
            \hline
            DSO               & Distribution System Operator   \\
            \hline
            DD                & Design Document                \\
            \hline
            WPX               & World Phenomena X              \\
            \hline
            SPX               & Shared Phenomena X             \\
            \hline
            GX                & Goal Number X                  \\
            \hline
            DAX               & Domain Assumptions X           \\
            \hline
            UCX               & Use Case X                     \\
            \hline
            EVD               & Electric Vehicle Driver        \\
            \hline
            EV                & Electric Vehicle               \\
            \hline
        \end{tabular}
        \caption{Acronyms used in the document.}
        \label{tab:acronyms}%
    \end{center}
\end{table}


\section{Reference Documents}
\label{sec:reference_documents}%
\begin{itemize}
    \item \href{https://polimi365-my.sharepoint.com/:b:/g/personal/10685242_polimi_it/EWPABzzjfF9EsgYvSiuvdAIBAz6qnjdfLuPE8kwQSxeyCg?e=6qasKD}{The specification document Assignment RDD AY 2022--2023.pdf}
\end{itemize}


\section{Document Structure}
\label{sec:doc_structure}%
The document is structured in seven sections, as described below.

First section introduce the purposes;
abbreviations and definitions useful to understand the problem are listed as well.

The following section, the second one, provides the chosen architectural design for the problem: here we describe
the identified system components, their relations, the offered communication interfaces, their behavior in the system
and architectural styles and design patterns used.

Later on, the third section focuses on user interfaces, presenting and describing the mockups offered to users.

The fourth section shows how the system meets the requirements.
At first, the section provides the mapping between identified components and functional requirements listed in the third
section of RASD\@.
Then, a description of the satisfaction of performance requirements and system attributes is provided.

Lastly, the fifth section provides a plan %TODO: add a quick description of the last section.

Section six reports the effort spent by each group member in the redaction of this document, meanwhile the last
section simply lists bibliography references and other resources used to redact this document.



    \chapter{Overall Description}
    \label{ch:overall_description}%
    \section{Product perspective}
\label{sec:product_perspective}%

\subsection{Class Diagrams}
\label{subsec:class_diagrams}%
The figure below lists and describes the classes involved in the system, their basic functionalities, their basic attributes,
and the relationships between them.
Some suggestions for a further expansion and deepening of the diagram below could be to evaluate the use of a decorator
pattern to implement the ``Fee'' class;
also, to evaluate the use of a status pattern to assign the state of a charging point (free, booked, occupied, broken).
Furthermore, another suggestion could be to adopt the factory pattern to implement the ``plug'' interface.
\begin{figure}[H]
    \begin{center}
        \includegraphics[width=0.9\linewidth]{ClassDiagrams/Class_Diagram_-_noPatterns}
        \caption{A simplified Class Diagram}
        \label{fig:class_diagram}%
    \end{center}
\end{figure}

\subsection{State Diagrams}
\label{subsec:state_diagrams}%

\paragraph{The EVD gets position and characteristics of charging stations at a certain location.}
EVD Andrew is going to use his car to go to the university for the Software Engineering 2 exam, but his EV is out of battery.
So, he needs to decide where to charge his vehicle.
To do that, he opens the \verb|eMALL| application and enters the map section.
At first, he sees if there is any charging station around him, but unfortunately at his current position,
there is only one charging station, which is shown as in maintenance.
So, he decides to see where to charge his EV nearby the university, inserting Milan in the location bar.
From the huge amount of charging stations, he decides to decide the one that costs less than the other ones.
So, he selects a charging station and gets its additional information.
He goes on searching other stations until he finds the best one for him.
At this point, the navigation process ends.

It is shown a state diagram that summaries the flow of activities done in the charging stations navigation process:
\begin{figure}[H]
    \begin{center}
        \includegraphics[width=0.8\linewidth]{StateDiagrams/get_charging_stations_location_state_diagram}
        \caption{Get locations of charging stations state diagram}
        \label{fig:locations_sd}%
    \end{center}
\end{figure}

\paragraph{EVD books a charge at a specified charging station at a certain timeframe.}
Andrew needs to book a charge for his EV\@.
He selects a charging station on the map and enters the booking section.
Unfortunately, the charging station cannot offer a reservation to him because of no availability status.
So, he searches for another one until he finds it.
Andrew has to decide in which timeframe he wants the charging point to be reserved.
So, he gets the availability schedule of the charging station and selects when he thinks to go to charge.
The system asks to pay a deposit to the EVD, which makes the payment.
Finally, the EVD receives an e-mail with all the information that confirms the reservation.

It is shown a state diagram that summaries the activities in the booking process:
\begin{figure}[H]
    \begin{center}
        \includegraphics[width=0.9\linewidth]{StateDiagrams/book_a_charge_state_diagram}
        \caption{Book a charge state diagram}
        \label{fig:booking_sd}%
    \end{center}
\end{figure}

\paragraph{CPO adds charging points in its CPMS.}
\verb|SOLARIS| is the new company of the successful businessman Hugh Peter.
They decided to trust the \verb|eMALL| project, entrusting them with the responsibility of managing their IT infrastructure.
After logging in, they start inserting new charging points owned by them and distributed throughout the territory.
The first thing they are asked to select is the charging station to which the new spot belongs.
So, they insert all the requested information (serial number, location, connectors, power, etc.).
After they confirm and submit what they inserted, they iterate the process until they have inserted all the charging points.
It is shown a state diagram to summaries the activities in the charging points insertion process:
\begin{figure}[H]
    \begin{center}
        \includegraphics[width=0.9\linewidth]{StateDiagrams/insert_charging_points_state_diagram}
        \caption{Insert charging points state diagram}
        \label{fig:insert_charging_points_sd}%
    \end{center}
\end{figure}

\subsection{Scenarios}
\label{subsec:scenarios}%

\paragraph{Unregistered EVD creates an account.}
Mr. Oak has his EV and is looking for an application that offers the chance to charge his vehicle and smartly plan a
charging process depending on the battery status and his daily schedule.
Fortunately, he finds out \verb|eMALL|\@.
So, he immediately proceeds to create an account.
At first, he opens the application and goes to the ``sign up'' section.
He inserts his name, second name, date of birth, living address, e-mail address, password, and telephone number.
He receives an e-mail with a 6-digit code to be inserted in the new window shown by the \verb|eMALL| system to confirm his e-mail address.
After accepting the terms \& conditions and submitting all the inserted information, the system creates his account,
and he can start using the application.

\paragraph{The EVD charges his/her EV.}
After booking a charge, Michael Scarn goes to the charging station at the chosen hour.
After turning off the EV, he opens the \verb|eMALL| application and enters the charge section.
From the set of close charging points, he selects which one has the serial number he received by \verb|eMALL| by e-mail
when he booked the charge.
So, he asks to charge the EV at that charging point.
After verifying that the EVD can charge at that charging point, the application communicates to the user that the
connectors are now unlocked and ready for charging his EV\@.
While the EV is in charge, the system notifies the EVD of the current status of the charging process.
When the process ends, he unplugs the connector, pays through the \verb|eMALL| application, and gets back in his car.

\paragraph{The EVD inserts a new activity in its calendar and receives a suggestion for a charging process.}
Jimothy inserts a new activity in his calendar, specifying the hour and destination of the event.
After doing that, he receives a notification that shows the EVD where and when to charge his vehicle.
The system creates suggestions to minimize the cost and the time lost at the charging station.
It also considers special offers activated by the CPOs registered in the eMALL system.
So, Joe accepts the received proposal and confirms the book of the listed charging points, making the needed payments.

\paragraph{The EVD receives a notification about a new special offer activated by a CPO}
Joe receives a notification about a new special offer activated by the CPO \verb|SOLARIS|.
So, he opens the promotion page, reads what it is about, and gets the discount code of the offer.
It consists of a 20\% discount for all the EVDs that are under 25.
Considering that he has to charge his EV, decides to book a charging session at a charging station owned by the CPO \verb|SOLARIS|.
After selecting the timeframe and verifying its availability, he inserts the discount code SARTORIUS\@.


\section{Product functions}
\label{sec:product_functions}%
\textbf{The EVD books a charging session} \\
The main functionality of the eMALL is to book charge sessions in different charging stations for the EVD\@.
In particular, the system shows charging stations to the EVD and waits for him to select where he wants to book a charging session.

When the eMALL retrieves information about the charging stations available in the local area, it also retrieves all the extra info about the available plugs and power supplies.

The system has to control if the charging station is currently unavailable, and if it is not, it gets the station's schedule.
The EVD has to choose a timeframe between the ones available to book a charge session.
The eMALL also queries the charging station to know if the station has or not a mandatory deposit to pay to end the booking process.
If the station does not have a deposit policy, then the eMALL finishes the booking process by sending an informative email to the EVD that resumes all the booking info.

In the email, the eMALL also specifies the serial number associated with the charging point of the charging station where the EVD has to charge his EV\@.

\textbf{The EVD receives charging alerts about where to charge his EV} \\
The eMALL offers smart functions about when the EVD might book a charge for his EV\@.
Hence, when an EVD inserts a new activity in his calendar, the eMALL computes the best route to reach the destination through an external navigator API\@.
The eMALL also checks the battery status of the EV, so it notifies the best itinerary for the EVD\@.
If the battery state doesn't allow the EVD to reach the destination, then the eMALL shows him the best route with the charging stations available along the road.

The eMALL tries to minimize the costs.
Hence, starting from the current battery status, the system computes the maximum kilometers an EV can travel before running out of battery.
If the EV can reach the destination, the eMALL marks the route returned by the API as preferred.
On the other hand, the eMALL finds charging stations along the road and selects the one with the minimum costs because it knows the details about the EV, for instance, the plug type.
The best charging station found is shown to the EVD, allowing him to decide whether to start a booking process.

If the EVD doesn't accept the eMALL solution, he can book another charging station along the road and start, as well, a booking process.

\textbf{The CPO manages its charging stations} \\
A CPO should be able to manage its charging stations and relative charging points.
In general, a CPO might have new charging stations to register in the eMALL, and, as well, it might have charging points too.
The system allows the CPO to register charging stations, by entering all the info about them, for instance, the position on the map and the number of charging points available.
Furthermore, the system allows the registration of also charging points by inserting info like the available plugs and the power supply of the charging process.

Just like the CPO inserts new information about its product, it can also delete charging points or charging stations from the eMALL\@.

The system also shows CPOs charging stations and relative charging points on the map.
This functionality is necessary because they might break down, so the CPO has to change their availability status (offline, under maintenance, online).


\section{User characteristics}
\label{sec:user_characteristics}%
The actors listed below are considered in the eMALL system
\begin{itemize}
    \item \textbf{CPO:} owns one or more charging stations, and manages bookings and promotions about its charging points.
    He buys energy from DSOs, based on prices and needs.
    CPOs has their own IT system.
    \item \textbf{Unregistered EV Driver:} anybody who owns an electric vehicle, but isn’t registered in the eMALL system.
    Before accessing its benefits, it needs to get an account.
    \item \textbf{Registered EV Driver:} an electric vehicle owner who already joined the eMALL system, and access its benefits.
    He’s identified with a unique ID, and can own one or more vehicles with different specifics.
    They can check prices and position of charging points, in addition to receiving notifications about promotions reserved to them.
\end{itemize}


\section{Assumptions, dependencies and constraints}
\label{sec:assumptions_dependencies_and_constraints}%
\newcounter{da}
\setcounter{da}{1}
\newcommand{\cda}{\theda\stepcounter{da}}
\begin{center}
    \begin{longtable}{ |l|p{0.9\linewidth}| }
        \hline
        \textbf{ID} & \textbf{Description}                                                                                     \\
        \hline
        DA\cda      & Each user has needed competences to use the \verb|eMALL| system.                                          \\
        \hline
        DA\cda      & The EVD has an e-mail.                                                                                   \\
        \hline
        DA\cda      & The EVD has a payment method.                                                                            \\
        \hline
        DA\cda      & The EVD uses a device with internet connectivity.                                                        \\
        \hline
        DA\cda      & The EVD uses a device with GPS module for navigation and localization.                                   \\
        \hline
        DA\cda      & There is a specific compatibility between EV's plug and connectors offered by charging points            \\
        \hline
        DA\cda      & Charging points have their own software.                                                                 \\
        \hline
        DA\cda      & Communication between the \verb|eMALL| system and the charging points happens through the OCPP protocol. \\
        \hline
        DA\cda      & The \verb|eMALL| system communicates with EV brands API to get vehicles' specification.                  \\
        \hline
        DA\cda      & The \verb|eMALL| system communicates with DSOs through their APIs.                                       \\
        \hline
        DA\cda      & The \verb|eMALL| system communicates with third-party payment services to manage the payments.           \\
        \hline
        \caption{Domain assumptions.}
        \label{tab:domainassmptn_tab}%
    \end{longtable}
\end{center}



    \chapter{Specific Requirements}
    \label{ch:specific_requirements}%
    \section{External Interface Requirements}
\label{sec:external_interface_requirements}%

\subsection{User Interfaces}
\label{subsec:user_interfaces}%
The eMALL’s user interfaces are a website and a mobile application;
the first is developed to be used mainly by CPOs with a dedicated login section for businesses but can be used by EVDs too.
The mobile application is available for Android and iOS and provides an enhanced experience as compared to the website
since it offers users personalized suggestions based on their location.
The website and the app should be easy to use since they will be used mostly by middle-aged users,
who might not always be familiar with the technology.
A “quick booking” section dedicated to facilitating the booking process might be included,
for those EVDs who are used to booking a charge at the same charging station (based on suggestions given by AI).

\subsection{Hardware Interfaces}
\label{subsec:hardware_interfaces}%
The system only requires a smartphone or computer with an internet connection and web browser to access websites or mobile applications.
Furthermore, eMALL communicates with the EV through its company's API to get the current battery level, the charging state,
so if it is plugged in and if it is charging, and the number of routable kilometers obtained on the current battery level.
To access personalized suggestions, based on EV’s position, the device in use has to be able to detect its location with a GPS or Glonass localization system.
%TODO: think id there is something you can write in this subsection

\subsection{Software Interfaces}
\label{subsec:software_interfaces}%

\subsection{Communication Interfaces}
\label{subsec:communication_interfaces}%
The eMALL system needs to communicate with other actors to offer functionalities to the users;
the communication is bidirectional and permits eMALL to obtain the desired data and serve elaborated data.
Below are listed different communication interfaces used to exchange information with users:
\begin{itemize}
    \item \textbf{CPMS and Charging Points.} The CPMSs offered to the CPO communicate with the charging point
    through the OCPP communication protocol.
    Thanks to it, the system can manage the charging session, given the possibility of starting and stopping it.
    Another significant functionality offered by OCPP is the diagnostic of the charging point:
    a CPO can reboot its charging spots, can get their log, and can update their firmware.
    %TODO: Does it make sense to put an example of an external system
    \item \textbf{eMALL and EVs.} The eMALL system communicates with the EVs registered by the EVD\@.
    As explained in the domain assumption section, we suppose that there is a third-party system that offers its API
    so to get the status of the battery of the EV. %An example of a system that provides these features is Smartcar,
    %which is already used by companies like AmpUp or BeCharge to remotely retrieve the battery level and remaining range of the EVs.
    \item \textbf{CPMS and DSOs.} The CPMSs offered to the CPO communicate with the DSOs through their APIs.
    CPOs can get selling prices set by the DSOs and they can decide from which DSO to acquire electricity.
    \item \textbf{eMALL and third-party payment services.} The \verb|eMALL| system offers the possibility to pay through
    external payment services to the EVD. The communication happens thanks to APIs offered by the companies that handle payments.
\end{itemize}


\section{Functional Requirements}
\label{sec:functional_requirements}%

\subsection{Requirements}
\label{subsec: requirements}%

%TODO: complete the table after having completed the use cases
\newcounter{req}
\setcounter{req}{1}
\newcommand{\creq}{\thereq\stepcounter{req}}
\begin{center}
    \begin{longtable}{|l|l|}
        \hline
        \textbf{ID} & \textbf{Description} \\
        \hline
        R\creq      &                      \\
        \hline
        \caption{Requirements.}
        \label{tab: req}%
    \end{longtable}
\end{center}

\subsection{Mapping on goals}
\label{subsec: map_on_g}%

%TODO: complete the table after having done the requirements
\newcounter{mg}
\setcounter{mg}{1}
\newcommand{\cmg}{\themg\stepcounter{mg}}
\begin{center}
    \begin{longtable}{|l|l|l|}
        \hline
        \textbf{Goal} & \textbf{Domain assumptions} & \textbf{Requirements} \\
        \hline
        G\cmg         &                             &                       \\
        \hline
        G\cmg         &                             &                       \\
        \hline
        G\cmg         &                             &                       \\
        \hline
        G\cmg         &                             &                       \\
        \hline
        G\cmg         &                             &                       \\
        \hline
        G\cmg         &                             &                       \\
        \hline
        G\cmg         &                             &                       \\
        \hline
        G\cmg         &                             &                       \\
        \hline
        G\cmg         &                             &                       \\
        \hline
        G\cmg         &                             &                       \\
        \hline
        G\cmg         &                             &                       \\
        \hline
        G\cmg         &                             &                       \\
        \hline
        \caption{Mapping on goals.}
        \label{tab: map_on_g}%
    \end{longtable}
\end{center}
%TODO: correct the diagrams if you add or remove use cases

\subsection{Use case diagrams}
\label{subsec: use_case_diag}%

\textbf{Unregistered EVD}
\begin{figure} [H]
    \begin{center}
        \includegraphics[width=0.9\linewidth]{Images/UseCaseDiagrams/unregistered_EVD_use_case_diagram}
        \caption{Unregistered EVD use case diagram}
        \label{fig: unreg_EVD_diag}
    \end{center}
\end{figure}

\textbf{Registered EVD}
\begin{figure} [H]
    \begin{center}
        \includegraphics[width=0.9\linewidth]{Images/UseCaseDiagrams/registered_EVD_use_case_diagram}
        \caption{Unregistered EVD use case diagram}
        \label{fig: reg_EVD_diag}
    \end{center}
\end{figure}

\textbf{CPO}
\begin{figure} [H]
    \begin{center}
        \includegraphics[width=0.9\linewidth]{Images/UseCaseDiagrams/CPO_use_case_diagram}
        \caption{Unregistered EVD use case diagram}
        \label{fig: cpo_diag}
    \end{center}
\end{figure}

\subsection{Use cases}
\label{subsec: use_cases}%
%TODO: write it better
In this section, they are explained and represented the main identified use cases.
There is a table with entry conditions, event flow, exit conditions and exception for each of them, and a sequence diagram
that shows the messages exchanged between the entities and the called functions.

\subsection{Mapping on requirements}
\label{subsec: map_on_req}%
\newcounter{mr}
\setcounter{mr}{1}
\newcommand{\cmr}{\themr\stepcounter{mr}}
\begin{center}
    \begin{longtable}{|l|l|}
        \hline
        \textbf{Use Case} & \textbf{Requirements} \\
        \hline
        \caption{Mapping on requirements.}
        \label{tab: map_on_req}
    \end{longtable}
\end{center}


\section{Performance Requirements}
\label{sec:performance_requirements}%
According to a market analysis conducted by \verb|MOTUS-E| in September 2022,
the number of fully electric vehicles and plug-in hybrid vehicles registered in Italy is 320.776.
If we suppose that the eMALL system will be used by one in every three EVDs,
the system should guarantee that it can handle simultaneously an overall of 100.000 connected clients, more or less.
\\
From the data storage point of view, the \verb|eMALL| system should consider three different sources of data:
\begin{itemize}
    \item \textbf{EVD's personal data.} We consider that $5\ KB$ is enough for the storage of personal information of an EVD\@.
    Considering $10^5$ EVDs, it is needed a total of:
    \begin{center}
        $10^5\cdot5\ KB = 488,3\ MB$
    \end{center}
    \item \textbf{EVD's calendar.} One of the functionalities offered by the \verb|eMALL| system is to insert new activities
    into EVD's calendar.
    The events have not much information: they specify starting time and destination of the activity.
    We can assume that each event requires $0.5 KB$ of storage.
    Considering all the potential users and assuming that they insert three activities a day, for the first year they are needed:
    \begin{center}
        $10^5\cdot 3\cdot 365\cdot 1\ KB = 104,43\ GB$
    \end{center}
    \item \textbf{History of charging sessions.} The \verb|eMALL| system should save the information of all the charging sessions.
    If we assume that the information of each charging session requires $3\ KB$ of storage.
    Considering all the potential users and assuming that they charge their vehicle once a day, for the first year they are needed:
    \begin{center}
        $10^5\cdot 365\cdot 3\ KB = 104,43\ GB$
    \end{center}
    %TODO: complete sources for data
\end{itemize}


\section{Design Constraints}
\label{sec:design_constraints}%

\subsection{Standards compliance}
\label{subsec:standards_compliance}%

\subsection{Hardware limitations}
\label{subsec:hardware_limitations}%

\subsection{Any other constraint}
\label{subsec:any_other_constraint}%


\section{Software System Attributes}
\label{sec:software_system_attributes}%

\subsection{Reliability}
\label{subsec:reliability}%

\subsection{Availability}
\label{subsec:availability}%

\subsection{Security}
\label{subsec:security}%

\subsection{Maintainability}
\label{subsec:maintainability}%

\subsection{Portability}
\label{subsec:portability}%



    \chapter{Formal Analysis Using Alloy}
    \label{ch:formal_analysis_using_alloy}%
    This section describes the model built to represent the world in which the \verb|eMALL| system works.

\section{Alloy Code}
\label{sec:alloy}%
\begin{lstlisting}[language=alloy,label={lst:alloy_code}]
open util/ordering[DateTime]

sig Appointment {
	startDate : DateTime,
	endDate : DateTime,
	chargingPoint : ChargingPoint
} {this in Calendar.appointments}

sig Battery {} {this in ChargingStation.batteries}

sig Calendar {appointments : disj set Appointment} {this in EVD.calendar}

sig ChargingPoint {
	eV : disj lone EV,
	plugs : some Plug
} {
	EV.plug in plugs and
	this in ChargingStation.chargingPoints
}

sig ChargingStation {
	chargingPoints : disj some ChargingPoint,
	batteries : disj set Battery,
	wayOfCharging : disj Battery + DSO
} {this in CPO.chargingStations}

sig CPO {
	chargingStations : disj some ChargingStation,
	dso : DSO
}

sig DateTime {} {this in Appointment.startDate + Appointment.endDate}

sig DSO {} {this in CPO.dso}

sig Email {} {this in EVD.email}

sig EV {plug : Plug} {this in UnregisteredEVD.eVs + EVD.eVs}

sig EVD {
	calendar : disj Calendar,
	email : disj Email,
	eVs : disj some EV,
	password : disj Password
}

sig Password {} {this in EVD.password}

abstract sig Plug {}
one sig CCS extends Plug {}
one sig ChaDeMo extends Plug {}
one sig Type1 extends Plug {}
one sig Type2 extends Plug {}

sig UnregisteredEVD {eVs : disj some EV}

/************************************************************************************/
/************************************************************************************/

///* An EVD cannot charge his EVs simultaneously, we assume each account is associated to only one driver
fact evdsCanChargeOnlyOneEvPerTime {
	all evd : EVD, disj ev1, ev2 : EV |
		ev1 + ev2 in evd.eVs and
		ev1 in ChargingPoint.eV implies
			ev2 not in ChargingPoint.eV
}
///* An appointment must start before ending
fact appointmentsAreCorrect {
	all a : Appointment |
		lt [a.startDate, a.endDate]
}
///* A booking process must not be overlapped to another booking process in the same calendar and in the same charging point
fact noOverlappedAppointmentsInChargingPointSchedules {
	no disj a1, a2 : Appointment |
		a1.chargingPoint in a2.chargingPoint and
		gte [a1.startDate, a2.startDate] and
		lte [a1.startDate, a2.endDate]
	no c : Calendar, disj a1, a2 : c.appointments |
		gte [a1.startDate, a2.startDate] and
		lte [a1.startDate, a2.endDate]
}
///* EVs of EVDs are not shared with unregistered EVDs
fact evsOfEvdsAreNotSharedWithUnregisteredEvds {
	all evd : EVD, uevd : UnregisteredEVD |
		#(evd.eVs & uevd.eVs) = 0
}
///* EVs of unregistered EVDs must not be connected to charging points
fact evsOfUnregisteredEvdsMustNotBeConnectedToChargingPoints {
	all cp : ChargingPoint, uevd : UnregisteredEVD |
		#(cp.eV & uevd.eVs) = 0
}
///* Charging stations charge vehicles through their batteries and DSOs
fact chargingStationsUseTheirBatteries {
	all cs : ChargingStation, cpo : CPO |
		cs in cpo.chargingStations and
		cs.wayOfCharging in cs.batteries + cpo.dso
}

/************************************************************************************/
/************************************************************************************/

///* EVs are connected to compatible charging points
assert evsAreConnectedToCompatibleChargingPoints {
	no cp : ChargingPoint |
		cp.eV.plug not in cp.plugs
}
///* No overlapped appointments in charging point schedules
assert noOverlappedAppointmentsInChargingPointSchedules {
	no disj a1, a2 : Appointment |
		a1.chargingPoint in a2.chargingPoint and
			lte [a1.startDate, a2.startDate] and
			lte [a1.endDate, a2.startDate]
}

/************************************************************************************/
/************************************************************************************/

///* Add new appointment to the calendar for an EVD
pred addNewAppointmentToCalendarForEvd [evd : EVD, a' : Appointment] {
	evd.calendar.appointments = evd.calendar.appointments + a'
}
///* Add new charging point to a charging station
pred addNewChargingPointToChargingStation [cs : ChargingStation, cp' : ChargingPoint] {
	cs.chargingPoints = cs.chargingPoints + cp'
}
///* Add new charging station to a CPO
pred addNewChargingStationToCpo [cpo : CPO, cs' : ChargingStation] {
	cpo.chargingStations = cpo.chargingStations + cs'
}
///* Add new EV to an EVD
pred addNewEvToEvd [evd : EVD, ev' : EV] {
	evd.eVs = evd.eVs + ev'
}
///* Add new plug in a charging point
pred addNewPlugToChargingPoint [cp : ChargingPoint, p' : Plug] {
	cp.plugs = cp.plugs + p'
}
///* Remove appointment from the calendar of an EVD
pred removeAppointmentFromCalendarOfEvd [evd : EVD, a : Appointment] {
	evd.calendar.appointments = evd.calendar.appointments - a
}
///* Remove a charging point from a charging station
pred removeChargingPointFromChargingStation [cs : ChargingStation, cp : ChargingPoint] {
	cs.chargingPoints = cs.chargingPoints - cp
}
///* Remove a charging station from a CPO
pred removeChargingStationFromCpo [cpo : CPO, cs : ChargingStation] {
	cpo.chargingStations = cpo.chargingStations - cs
}
///* Remove EV from an EVD
pred removeEvFromEvd [evd : EVD, ev : EV] {
	evd.eVs = evd.eVs - ev
}
///* Remove plug from a charging point
pred removePlugFromChargingPoint [cp : ChargingPoint, p : Plug] {
	cp.plugs = cp.plugs - p
}
///* Update email in an EVD
pred updateEmailInEvd [evd : EVD, e' : Email] {
	evd.email = e'
}
///* Create a simple world
pred simpleWorld {
	#ChargingStation = 1
	#ChargingPoint = 3
	#EVD = 2
	#Appointment = 2
}
///* Create a world where there are many appointments
pred worldWithManyAppointments {
	#ChargingStation = 1
	#ChargingPoint = 4
	#EVD = 2
	#Appointment = 6
}

/************************************************************************************/
/************************************************************************************/

run addNewAppointmentToCalendarForEvd

run addNewChargingPointToChargingStation

run addNewChargingStationToCpo

run addNewEvToEvd

run addNewPlugToChargingPoint

run removeAppointmentFromCalendarOfEvd

run removeChargingPointFromChargingStation

run removeChargingStationFromCpo

run removeEvFromEvd

run removePlugFromChargingPoint

run updateEmailInEvd

run simpleWorld

run worldWithManyAppointments for 10

check evsAreConnectedToCompatibleChargingPoints

check noOverlappedAppointmentsInChargingPointSchedules
\end{lstlisting}

\section{Simulations}
\label{sec: sim}%
In this section we show to simulations of the built model.
The first one is a simple world, with few instances, useful to understand the basis of the relations between entities.
The second world is more complex due to the representation of a higher number of instances of stations, EVDs and appointments.

\begin{sidewaysfigure}
	\begin{figure} [H]
		\begin{center}
			\includegraphics[width=1\linewidth]{graphs/simpleWorld}
			\caption{Simple world Alloy.}
			\label{fig: simple_world_alloy}
		\end{center}
	\end{figure}
\end{sidewaysfigure}

\begin{sidewaysfigure}
	\begin{figure} [H]
		\begin{center}
			\includegraphics[width=1\linewidth]{graphs/worldWithManyAppointments}
			\caption{Simple world Alloy.}
			\label{fig: many_appointments_world_alloy}
		\end{center}
	\end{figure}
\end{sidewaysfigure}



    \chapter{Effort Spent}
    \label{ch:effort_spent}%
    \begin{table}
    \begin{center}
        \caption{Effort spent by each member of the group.}
        \label{tab:effor_spent}
        \begin{tabular}{c|c}
            Member of group & Effort spent \\
            Cela Irfan & \begin{tabular}{c|c}
                             Introduction          & $h$ \\
                             Overall description   & $h$ \\
                             Specific requirements & $h$ \\
                             Formal analysis       & $h$ \\
                             Reasoning             & $h$ \\
            \end{tabular} \\
            Cela Mario & \begin{tabular}{c|c}
                             Introduction          & $h$ \\
                             Overall description   & $h$ \\
                             Specific requirements & $h$ \\
                             Formal analysis       & $h$ \\
                             Reasoning             & $h$ \\
            \end{tabular} \\
            Cogollo Alessandro & \begin{tabular}{c|c}
                                     Introduction          & $h$ \\
                                     Overall description   & $h$ \\
                                     Specific requirements & $h$ \\
                                     Formal analysis       & $h$ \\
                                     Reasoning             & $h$ \\
            \end{tabular} \\
        \end{tabular}
    \end{center}
\end{table}



    \chapter{References}
    \label{ch:references}%
    \section{Paper references}
\label{sec:paper_references}%
\begin{itemize}
    \item \href{https://polimi365-my.sharepoint.com/:b:/g/personal/10685242_polimi_it/EWPABzzjfF9EsgYvSiuvdAIBAz6qnjdfLuPE8kwQSxeyCg?e=6qasKD}{The specification document Assignment RDD AY 2022-2023.pdf}
    \item \href{https://www.platformelectromobility.eu/2022/05/17/ev-charging-how-to-tap-in-the-grid-smartly/}{Platform for Electromobility. EV Charging: How to tap in the grid smartly?}
    \item \href{https://mobilityintegrationsymposium.org/wp-content/uploads/sites/10/2018/11/4A_3_Emob18_024_paper_Filipe_Campos.pdf}{F. Campos, L. Marques, and K. Kotsalos, Electric Vehicle CPMS and Secondary Substation Management. 2nd E-Mobility Power System Integration Symposium,  Stockholm,  Sweden,  15 October 2018. }
    \item \href{https://polimi365-my.sharepoint.com/:b:/g/personal/10685242_polimi_it/EfCXzQWCkK1Lsdtr4suEMp8B7YR3drdGqkArs7hnEx-bqA?e=xy1OTu}{Shu Su, Hui Yan, and Ning Ding. 2018. Machine Learning-Based Charging Network Operation Service Platform Reservation Charging Service System}
    \item \href{https://www.motus-e.org/analisi-di-mercato/settembre-2022-troppa-incertezza-consumatori-preoccupati-non-acquistano/}{September 2022 Market Analysis - MOTUS-E}
\end{itemize}


\section{Used tools}
\label{sec:used_tools}%
\begin{itemize}
    \item \href{https://github.com/}{GitHub} for project versioning
    \item \href{https://staruml.io/}{StarUML} for UML diagrams
    \item \href{https://www.notion.so/}{Notion} for reasoning and notes
    \item \href{https://www.jetbrains.com/idea/}{IntelliJ} as \LaTeX\ editor
    \item \href{https://alloytools.org/}{Alloy} for formal analysis
    \item \href{https://code.visualstudio.com/}{Visual Studio Code} as Alloy editor
\end{itemize}



%-------------------------------------------------------------------------
%	APPENDICES
%-------------------------------------------------------------------------

    \cleardoublepage
    \addtocontents{toc}{\vspace{2em}} % Add a gap in the Contents, for aesthetics
    \appendix


    \chapter{Appendix A}
    \label{ch:appendix_a}%
    If you need to include an appendix to support the research in your thesis, you can place it at the end of the manuscript.
    An appendix contains supplementary material (figures, tables, data, codes, mathematical proofs, surveys, \dots)
    which supplement the main results contained in the previous chapters.


% LIST OF FIGURES
    \listoffigures

% LIST OF TABLES
    \listoftables
    \cleardoublepage
\end{document}
