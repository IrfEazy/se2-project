The EVs are eco-friendly vehicles that will be on our roads in the next future.
In order to keep global warming below 1.5°C, Europe have decided to reduce greenhouse gas emissions of CO2 per
person per year by 2030, and, by the same year, the IEA predicts that electric vehicles will have a market share
of roughly 30 percent, with a total number of 23 million e-cars on the roads.
EVs consumption is measured in kilowatt-hours per 100 kilometers, and most of the current electric cars can travel
between 150 and 350 kilometers on a single charge, but premium-brand models can currently cover more than 500
kilometers.

In this context, when people use an electric vehicle, knowing where to charge it and carefully planning the
charging process in such a way that it introduces minimal interference and constraints on our daily schedule
is of great importance.

That's were \verb|eMALL| operates: it can find charging stations owned by several Charging Point Operators - CPO - and,
considering the activities in user's schedule, it can propose the best possible path of charging process
in order to minimize the cost and the waisted time at the station.


\section{Purpose}
\label{sec:purpose}
\newcounter{g}
\setcounter{g}{1}
\newcommand{\cg}{\theg\stepcounter{g}}
\begin{table}[H]
    \centering
    \begin{tabular}{ |l|p{0.9\linewidth}| }
        \hline
        \textbf{ID} & \textbf{Description}                                                                                                     \\
        \hline
        G\cg        & The EVD can see charging stations nearby a specific location on the map                                                  \\
        \hline
        G\cg        & The EVD can get the costs of charging stations                                                                           \\
        \hline
        G\cg        & The EVD can search for special offer provided by charging stations                                                       \\
        \hline
        G\cg        & The EVD can book a charge for his EV at a charging station for a specified time frame                                    \\
        \hline
        G\cg        & The EVD can pay for the recharging service                                                                               \\
        \hline
        G\cg        & Given the destination inserted in an activity in his calendar application, the EVD receives suggestions to charge his EV \\ %TODO: not correctly explained
        \hline
    \end{tabular}
    \caption{The goals.}
    \label{tab:goals_tab}
\end{table}


\section{Scope}
\label{sec:scope}


\section{Definition, Acronyms, Abbreviations}
\label{sec:definition_acronyms_abbreviations}
\begin{table}[H]
    \begin{center}
        \begin{tabular}{ |l|l| }
            \hline
            \textbf{Acronyms} & \textbf{Definition}                              \\
            \hline
            eMSP              & e-Mobility Service Provider                      \\
            \hline
            CPO               & Charging Point Operator                          \\
            \hline
            CPMS              & Charge Point Management System                   \\
            \hline
            DSO               & Distribution System Operator                     \\
            \hline
            RASD              & Requirements Analysis and Specification Document \\
            \hline
            WP                & World Phenomena                                  \\
            \hline
            SP                & Shared Phenomena                                 \\
            \hline
            GX                & Goal Number X                                    \\
            \hline
            EVD               & Electric Vehicle Driver                          \\
            \hline
        \end{tabular}
        \caption{Acronyms used in the document.}
        \label{tab:acronyms}
    \end{center}
\end{table}


\section{Revision history}
\label{sec:revision_history}


\section{Reference Documents}
\label{sec:reference_documents}
The specification document \verb|Assignment RDD AY 2022-2023.pdf|.


\section{Document Structure}
\label{sec:document_structure}
The document is structured in six sections, as described below.

First section introduce the goals of the project, purposes, and a brief analysis on world and shared phenomena;
abbreviations and definitions useful to understand the problem are listed as well.

The following section, the second one, provides an overall description of the problem: here scenarios and further
details on domain, and scenarios are included, aside from more product and user characteristics, assumptions,
dependencies and constraints.

Later on, the third section focuses on the specific requirements and provides a more detailed analysis of external
interface requirements, functional requirements and performance requirements.

Lastly, the fourth section provides a formal analysis, using alloy.
This chapter is crucial to prove the correctness of the model described in the previous sections, and should focus on
reporting results of the checks performed and meaningful assertions.

Section five reports the effort spent by each group member in the redaction of this document, meanwhile the last
section simply lists bibliography references and other resources used to redact this document.
